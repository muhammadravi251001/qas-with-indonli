%-----------------------------------------------------------------------------%
\chapter{\babLima}
\label{bab:5}
%-----------------------------------------------------------------------------%
Bab kelima ini menjelaskan tentang hasil dan analisis eksperimen yang penulis lakukan dalam penelitian ini. Sub-bab 5.1 menjelaskan mengenai hasil dan analisis eksperimen metode \emph{Intermediate-Task Transfer Learning}. Sub-bab 5.2 menjelaskan mengenai hasil dan analisis eksperimen metode \emph{Task Recasting} dengan IndoNLI sebagai verifikator. SUb-bab 5.3 menjelaskan mengenai hasil dan analisis eksperimen metode \emph{Task Recasting} dengan IndoNLI sebagai penghasil jawaban.

%-----------------------------------------------------------------------------%
\section{Hasil dan Analisis Performa \emph{Intermediate-Task Transfer Learning}}
%-----------------------------------------------------------------------------%
Pada sub-bab ini, akan dijelaskan mengenai hasil dan analisis eksperimen metode \emph{Intermediate-Task Transfer Learning} yang penulis lakukan pada pada penelitian ini. 

%-----------------------------------------------------------------------------%
\subsection{Performa pada  \emph{dataset} SQuAD-ID}
%-----------------------------------------------------------------------------%

%-----------------------------------------------------------------------------%
\subsection{Performa pada  \emph{dataset} TyDi-QA-ID}
%-----------------------------------------------------------------------------%

%-----------------------------------------------------------------------------%
\subsection{Performa pada  \emph{dataset} IDK-MRC}
%-----------------------------------------------------------------------------%


%-----------------------------------------------------------------------------%
\section{Hasil dan Analisis Performa \emph{Task Recasting} dengan IndoNLI sebagai verifikator}
%-----------------------------------------------------------------------------%
Pada sub-bab ini, akan dijelaskan mengenai hasil dan analisis eksperimen metode \emph{Task Recasting} dengan IndoNLI sebagai verifikator yang penulis lakukan pada pada penelitian ini. 

%-----------------------------------------------------------------------------%
\subsection{Performa pada  \emph{dataset} SQuAD-ID}
%-----------------------------------------------------------------------------%

%-----------------------------------------------------------------------------%
\subsection{Performa pada  \emph{dataset} TyDi-QA-ID}
%-----------------------------------------------------------------------------%

%-----------------------------------------------------------------------------%
\subsection{Performa pada  \emph{dataset} IDK-MRC}
%-----------------------------------------------------------------------------%

%-----------------------------------------------------------------------------%
\section{Hasil dan Analisis Performa \emph{Task Recasting} dengan IndoNLI sebagai penghasil jawaban}
%-----------------------------------------------------------------------------%
Pada sub-bab ini, akan dijelaskan mengenai hasil dan analisis eksperimen metode \emph{Task Recasting} dengan IndoNLI sebagai penghasil jawaban yang penulis lakukan pada pada penelitian ini. 

%-----------------------------------------------------------------------------%
\subsection{Performa pada  \emph{dataset} SQuAD-ID}
%-----------------------------------------------------------------------------%

%-----------------------------------------------------------------------------%
\subsection{Performa pada  \emph{dataset} TyDi-QA-ID}
%-----------------------------------------------------------------------------%

%-----------------------------------------------------------------------------%
\subsection{Performa pada  \emph{dataset} IDK-MRC}
%-----------------------------------------------------------------------------%+