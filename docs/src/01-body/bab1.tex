%-----------------------------------------------------------------------------%
\chapter{\babSatu}
\label{bab:1}
%-----------------------------------------------------------------------------%
Bab pertama ini menjelaskan mengenai pendahuluan penelitian yang telah penulis rancang untuk penelitian ini. Sub-bab 1.1 menjelaskan mengenai latar belakang penelitian. Sub-bab 1.2 menjelaskan mengenai rumusan masalah yang akan dipecahkan pada penelitian ini. Sub-bab 1.3 menjelaskan tentang tujuan penelitian. Sub-bab 1.4 menjelaskan tentang manfaat penelitian. Sub-bab 1.4 menjelaskan tentang ruang lingkup dan limitasi dari penelitian ini. Sub-bab 1.6 menjelaskan tentang sistematika penulisan laporan penelitian ini.

%-----------------------------------------------------------------------------%
\section{Latar Belakang}
%-----------------------------------------------------------------------------%
Sistem tanya jawab (\emph{question answering}) merupakan suatu permasalahan pada domain \emph{natural language processing} (NLP), sederhananya permasalahan sistem tanya jawab adalah suatu \emph{task} di mana mesin akan belajar untuk menjawab pertanyaan yang diberikan oleh pengguna berdasarkan konteks yang ada \citep{stroh2016question}. Sehingga, dapat disimpulkan bahwa masukan dari \emph{task} ini adalah pertanyaan dan konteks; dan keluarannya adalah jawaban berdasarkan pertanyaan dan konteks yang diberikan sebelumnya. Pada \emph{task} sistem tanya jawab, kita dapat meningkatkan performa model dengan berbagai macam metode, salah satunya dengan memanfaatkan \emph{natural language inference} (NLI). NLI merupakan suatu \emph{semantic task} pada domain NLP yang bertujuan untuk mengkarakterisasi dan memanfaatkan relasi antar dua kalimat, untuk menyelesaikan hal tersebut, model membutuhkan kemampuan untuk mengurai semantik kalimat dan penalaran logis (\emph{commonsense}) yang baik \citep{bowman-etal-2015-large}. Sebenarnya, model sistem tanya jawab sudah banyak ditemui dengan berbasis bahasa Inggris, namun masih sedikit model sistem tanya jawab berbasis bahasa Indonesia dan juga performa model sistem tanya jawab berbahasa Indonesia masih kurang baik.

Berdasarkan uraian pada di atas, penulis berusaha untuk mengembangkan model sistem tanya jawab berbahasa Indonesia dengan memanfaatkan IndoNLI sebagai media penyelesaian jawaban, verifikator jawaban, dan peningkatan performa model sistem tanya jawab itu sendiri. IndoNLI merupakan \emph{dataset} NLI berbahasa Indonesia \citep{mahendra-etal-2021-indonli}. Pemanfaatan IndoNLI pada sistem tanya jawab berbahasa Indonesia rencananya akan menggunakan dua metode, yaitu: \emph{intermediate pre-training} dan \emph{task recasting}. \emph{Intermediate pre-training} merupakan teknik peningkatan performa dengan melakukan \emph{pre-training} dengan \emph{dataset} IndoNLI pada model BERT \emph{baseline}, lalu model tersebut akan dilakukan \emph{fine-tuning} untuk \emph{task} sistem tanya jawab-nya. Sedangkan \emph{task recasting} merupakan teknik peningkatan performa dengan melakukan \emph{recasting input} \& \emph{output} sistem tanya jawab menjadi \emph{input \& output task} NLI atau sebaliknya, hal tersebut dilakukan dalam rangka validasi hasil prediksi jawaban dari sistem tanya jawab maupun sebagai penghasil jawaban dari \emph{dataset} IndoNLI. Harapan penulis, dengan memanfaatkan IndoNLI, performa sistem tanya jawab berbahasa Indonesia dapat meningkat dibanding sistem tanya jawab sebelumnya yang tidak memanfaatkan IndoNLI.

%-----------------------------------------------------------------------------%
\section{Rumusan Masalah}
%-----------------------------------------------------------------------------%
Rumusan masalah yang ingin penulis coba selesaikan pada penelitian ini berdasarkan latar belakang yang sebelumnya disebutkan, adalah:

\begin{itemize}

    \item Apakah pemanfaatan IndoNLI dapat meningkatkan performa sistem tanya jawab berbahasa Indonesia?

    \item Jenis pertanyaan apa dalam sistem tanya jawab berbahasa Indonesia yang lebih baik ditangani dengan pemanfaatan IndoNLI?
    
\end{itemize}

%-----------------------------------------------------------------------------%
\section{Tujuan Penelitian}
%-----------------------------------------------------------------------------%
Tujuan yang ingin penulis capai dalam penelitian ini berdasarkan latar belakang yang sebelumnya disebutkan, adalah:

\begin{itemize}

    \item Merancang metode pemanfaatan IndoNLI pada sistem tanya jawab berbahasa Indonesia.

    \item Menganalisis peningkatan atau penurunan performa sistem tanya jawab berbahasa Indonesia jika ditambahkan model IndoNLI.

    \item Menganalisis jenis-jenis pertanyaan yang bisa dijawab dengan lebih baik oleh sistem tanya jawab dengan menerapkan model IndoNLI.
    
\end{itemize}

%-----------------------------------------------------------------------------%
\section{Manfaaat Penelitian}
%-----------------------------------------------------------------------------%
Melalui penelitian terkait sistem tanya jawab bahasa Indonesia dengan memanfaatkan IndoNLI ini, penulis berharap performa sistem tanya jawab berbahasa Indonesia dapat meningkat dibanding sistem tanya jawab sebelumnya yang tidak memanfaatkan IndoNLI. Sehingga, lebih banyak memantik topik-topik riset terkait NLP dengan tugas (\emph{task}) berbahasa Indonesia.

%-----------------------------------------------------------------------------%
\section{Ruang Lingkup dan Batasan Penelitian}
%-----------------------------------------------------------------------------%
Ruang lingkup dan batasan penelitian ini mencakup beberapa hal, antara lain:

\begin{itemize}

    \item Penelitian ini akan difokuskan kepada pencarian metode yang paling tepat guna memanfaatkan model IndoNLI pada sistem tanya jawab bahasa Indonesia, maka model yang nanti akan dieksperimenkan adalah model sistem tanya jawab bahasa Indonesia-nya, bukan model IndoNLI.

    \item Fokus tujuan penelitian ini adalah untuk mengetahui signifikansi penggunaan model IndoNLI pada sistem tanya jawab berbahasa Indonesia, dan juga untuk mengetahui kasus seperti apa (dalam konteks ini adalah jenis pertanyaan) yang lebih baik ditangani oleh pemanfaatan model IndoNLI pada sistem tanya jawab berbahasa Indonesia.
    
\end{itemize}

%-----------------------------------------------------------------------------%
\section{Sistematika Penulisan}
%-----------------------------------------------------------------------------%
Laporan ini diawali dengan abstrak yang terdiri dari ringkasan keseluruhan isi penelitian, pernyataan masalah, metode yang digunakan, analisis dan eksperimen dari hasil penelitian, dan kesimpulan penelitian; daftar isi, daftar gambar, dan daftar tabel. Kemudian, laporan dilanjut ke bab-bab penyusun, antara lain adalah, Bab I yang berisi bagian pendahuluan, Bab II yang berisi bagian studi literatur, Bab III yang berisi bagian metodologi, Bab IV yang berisi analisis dan eksperimen, dan Bab V yang berisi bagian penutup. 

Kemudian, pada Bab I akan membahas hal-hal yang berkaitan tentang latar belakang topik penelitian, rumusan masalah, tujuan penelitian, manfaat penelitian, ruang lingkup penelitian, batasan penelitian, dan sistematika penulisan. Lalu, pada Bab II akan membahas hal-hal yang berkaitan tentang studi literatur yang berkaitan dengan topik penelitian yang juga akan menjadi landasan teori penelitian topik sistem tanya jawab berbahasa Indonesia ini. Kemudian, pada Bab III akan membahas hal-hal yang berkaitan tentang metodologi yang digunakan pada penelitian ini. Lalu, pada Bab IV akan membahas hal-hal yang berkaitan tentang analisis dan eksperimen dari hasil penelitian yang telah penulis lakukan. Terakhir, pada Bab V penulis akan memberikan kesimpulan dan saran dari penelitian yang telah penulis lakukan terhadap topik sistem tanya jawab berbahasa Indonesia ini.