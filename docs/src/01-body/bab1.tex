%-----------------------------------------------------------------------------%
\chapter{\babSatu}
\label{bab:1}
%-----------------------------------------------------------------------------%
Bab pertama ini menjelaskan mengenai pendahuluan penelitian yang telah penulis rancang untuk penelitian ini. Sub-bab \ref{1.1} menjelaskan mengenai latar belakang penelitian. Sub-bab \ref{1.2} menjelaskan mengenai rumusan masalah yang akan dipecahkan pada penelitian ini. Sub-bab \ref{1.3} menjelaskan tentang tujuan penelitian. Sub-bab \ref{1.4} menjelaskan tentang manfaat penelitian. Sub-bab \ref{1.5} menjelaskan tentang ruang lingkup dan limitasi dari penelitian ini. Sub-bab \ref{1.6} menjelaskan tentang sistematika penulisan laporan penelitian ini.

%-----------------------------------------------------------------------------%
\section{Latar Belakang}
\label{1.1}
%-----------------------------------------------------------------------------%
Model sistem tanya jawab yang memiliki performa yang baik sebenarnya sudah banyak ditemui dengan berbasis bahasa Inggris, namun masih sedikit model sistem tanya jawab berbasis bahasa Indonesia dan juga performa model sistem tanya jawab berbahasa Indonesia masih kurang baik, penyebab hal tersebut terjadi mungkin saja karena masih sedikitnya riset terkait sistem tanya jawab di Indonesia. Contoh dari model sistem tanya jawab berbasis bahasa Indonesia adalah: dengan menggunakan pendekatan algoritma \emph{machine learning}, khususnya SVM \citep{machine-learning-approach}. Kemudian, ada juga perancangan sistem tanya jawab berbahasa Indonesia khusus pertanyaan \emph{non-factoid} \citep{Purwarianti_Yusliani_2012}. Terakhir, pada penelitian selanjutnya menggunakan metode \emph{knowledge graph} \citep{JLK}.

Berikut beberapa hal-hal yang ingin penulis perbaiki dari beberapa penelitian sistem tanya jawab yang sudah disebutkan di atas, antara lain: pada pendekatan algoritma \emph{machine learning} \citep{machine-learning-approach} masih memiliki performa yang kurang baik, dengan ditandai nilai akurasi terbaiknya masih di sekitar 0.66, dan juga masih membutuhkan klasifikasi pertanyaan (mirip seperti \emph{named-entity recognition}), dimana model klasifikasi pertanyaan berbahasa Indonesia masih memiliki performa yang kurang baik setidaknya hingga saat ini, lalu, pada penelitian sistem tanya jawab berbahasa Indonesia khusus pertanyaan \emph{non-factoid} \citep{Purwarianti_Yusliani_2012} juga masih memiliki performa yang kurang baik, dengan ditandai nilai MRR terbaiknya masih di sekitar 0.77, salah satu penyebab ketidakakuratan ini dijelaskan oleh \citeauthor{Purwarianti_Yusliani_2012}, bahwa: paragraf yang mengandung jawaban memiliki kata kunci yang tersebar pada beberapa kalimat dari paragraf tersebut sehingga nilai kalimat menjadi lebih rendah dibanding kalimat lain yang memiliki kata kunci yang lebih banyak; terakhir, pada penelitian sistem tanya jawab dengan menggunakan metode \emph{knowledge graph} \citep{JLK} mendapatkan skor F1 yang cukup tinggi, yaitu sebesar 0.85 dengan menggunakan metode \emph{knowledge graph} \citep{JLK}. Namun, penggunaan \emph{dataset} sistem tanya jawab yang terlalu terisolasi (hanya membahas terkait produk kecantikan) membuat skor F1 yang tinggi ini tidak dapat digeneralisir kepada tema-tema pertanyaan yang lebih luas, dan juga konstruksi \emph{query} yang digunakan sebagai basis dari \emph{knowledge graph} sulit untuk digeneralisasikan kepada konteks di luar dari konteks transaksi barang.

Penelitian ini didasari oleh hipotesis penulis, bahwa: pada \emph{task} sistem tanya jawab, ada kemungkinan bahwa kita dapat meningkatkan performa model dengan memanfaatkan \emph{natural language inference} (NLI), hipotesis tersebut didasari oleh keandalan NLI dalam mendeteksi relasi antar dua kalimat, yaitu: kalimat premis dan kalimat hipotesis. Hal tersebut membuat penulis berpikir bahwa NLI juga dapat dimanfaatkan untuk mendeteksi konsistensi ataupun kontradiksi antara jawaban dan konteks yang telah diberikan \citep{bowman-etal-2015-large, maccartney-manning-2009-extended, giampiccolo-etal-2007-third}. NLI merupakan suatu \emph{semantic task} pada domain NLP yang bertujuan untuk mengkarakterisasi dan memanfaatkan relasi antar dua kalimat, untuk menyelesaikan hal tersebut, model membutuhkan kemampuan untuk mengurai semantik kalimat dan penalaran logis (\emph{commonsense}) yang baik \citep{bowman-etal-2015-large}.

Dari uraian di atas, penulis berusaha menggali potensi NLI agar bisa meningkatkan performa sistem tanya jawab berbahasa Indonesia dengan merancang sistem tanya jawab berbahasa Indonesia dengan memanfaatkan NLI dengan tema \emph{dataset} yang luas, oleh karena itu \emph{dataset} sistem tanya jawab yang penulis gunakan pada penelitian ini merupakan \emph{dataset} yang telah menjadi tolak ukur (\emph{benchmark}) dari sistem tanya jawab, yaitu: SQuAD-ID, TyDi-QA-ID, dan IDK-MRC. Pada penelitian ini, penulis menggunakan tiga \emph{dataset} sekaligus, agar rancangan sistem tanya jawab ini dapat tergeneralisasi kepada tema-tema yang jauh lebih luas. Kemudian, pemanfaatan NLI juga dirasa mudah untuk diimplementasikan (dibandingkan pendekatan pada penelitian sebelumnya), apalagi sudah ada \emph{dataset} NLI berbahasa Indonesia yang dianotasikan langsung oleh manusia, yaitu: IndoNLI \citep{mahendra-etal-2021-indonli}.

 Sistem tanya jawab (\emph{question answering}) merupakan suatu permasalahan pada domain \emph{natural language processing} (NLP), sederhananya permasalahan sistem tanya jawab adalah suatu \emph{task} di mana mesin akan belajar untuk menjawab pertanyaan yang diberikan oleh pengguna berdasarkan konteks yang ada \citep{stroh2016question}. Sehingga, dapat disimpulkan bahwa masukan dari \emph{task} ini adalah pertanyaan dan konteks; dan keluarannya adalah jawaban berdasarkan pertanyaan dan konteks yang diberikan sebelumnya. 

Kemudian, metode yang penulis gunakan pada penelitian ini untuk meningkatkan performa sistem tanya jawab berbahasa Indonesia secara garis besar terbagi dua, yaitu: \emph{intermediate task transfer learning} dan \emph{task recasting}. Metode-metode tersebut secara langsung memanfaatkan NLI, khususnya IndoNLI \citep{mahendra-etal-2021-indonli}; karena IndoNLI merupakan \emph{dataset} NLI berbahasa Indonesia. Sederhananya, metode \emph{intermediate task transfer learning} adalah penerapan pengetahuan (\emph{knowledge}) dari \emph{task} NLI kepada \emph{task} sistem tanya jawab yang akan dirancang \citep{pruksachatkun-etal-2020-intermediate}. Lalu, metode \emph{task recasting} merupakan konversi \emph{task} sistem tanya jawab menjadi \emph{task} NLI \citep{chen-etal-2021-nli-models}.

Pada penelitian ini, juga akan dianalisis jenis-jenis pertanyaan apa saja yang  lebih baik ditangani dengan memanfaatkan NLI, khususnya IndoNLI. Jenis-jenis pertanyaan yang penulis analisis berdasarkan dua aspek besar, yaitu: \emph{length-based analysis} dan \emph{type-based analysis}. Hal tersebut penulis lakukan untuk mendapatkan wawasan (\emph{insight}) yang lebih dalam, tidak hanya sekadar meningkatkan performa sistem tanya jawab saja. Harapan penulis, dengan memanfaatkan NLI, performa sistem tanya jawab berbahasa Indonesia dapat meningkat dibanding sistem tanya jawab sebelumnya yang tidak memanfaatkan NLI dan juga dapat mengetahui jenis-jenis pertanyaan yang lebih baik ditangani dengan memanfaatkan NLI.

%-----------------------------------------------------------------------------%
\section{Rumusan Masalah}
\label{1.2}
%-----------------------------------------------------------------------------%
Rumusan masalah yang ingin penulis coba selesaikan pada penelitian ini berdasarkan latar belakang yang sebelumnya disebutkan, adalah:

\begin{itemize}

    \item Apakah pemanfaatan NLI dapat meningkatkan performa sistem tanya jawab berbahasa Indonesia?

    \item Jenis pertanyaan apa dalam sistem tanya jawab berbahasa Indonesia yang lebih baik ditangani dengan pemanfaatan NLI?
    
\end{itemize}

%-----------------------------------------------------------------------------%
\section{Tujuan Penelitian}
\label{1.3}
%-----------------------------------------------------------------------------%
Tujuan yang ingin penulis capai dalam penelitian ini berdasarkan latar belakang yang sebelumnya disebutkan, adalah:

\begin{itemize}

    \item Merancang metode pemanfaatan NLI pada sistem tanya jawab berbahasa Indonesia.

    \item Menganalisis peningkatan atau penurunan performa sistem tanya jawab berbahasa Indonesia jika dengan memanfaatkan NLI.

    \item Menganalisis karakteristik pasangan konteks-pertanyaan-jawaban yang bisa dijawab dengan lebih baik oleh sistem tanya jawab dengan memanfaatkan NLI.
    
\end{itemize}

%-----------------------------------------------------------------------------%
\section{Manfaat Penelitian}
\label{1.4}
%-----------------------------------------------------------------------------%
Melalui penelitian terkait sistem tanya jawab berbahasa Indonesia dengan memanfaatkan NLI ini, penulis berharap performa sistem tanya jawab berbahasa Indonesia dapat meningkat dibanding sistem tanya jawab sebelumnya yang tidak memanfaatkan NLI. Sehingga, lebih banyak memantik topik-topik riset terkait NLP dengan tugas (\emph{task}) berbahasa Indonesia.

%-----------------------------------------------------------------------------%
\section{Ruang Lingkup dan Batasan Penelitian}
\label{1.5}
%-----------------------------------------------------------------------------%
Ruang lingkup dan batasan penelitian ini mencakup beberapa hal, antara lain:

\begin{itemize}

    \item Penelitian ini akan difokuskan kepada pencarian metode yang paling tepat guna memanfaatkan NLI pada sistem tanya jawab bahasa Indonesia, maka model yang nanti akan dieksperimenkan adalah model sistem tanya jawab bahasa Indonesia-nya, bukan model NLI-nya.

    \item Fokus tujuan penelitian ini adalah untuk mengetahui peningkatan dengan memanfaatkan NLI pada sistem tanya jawab berbahasa Indonesia, dan juga untuk mengetahui kasus seperti apa (dalam konteks ini adalah karakteristik pasangan konteks-pertanyaan-jawaban) yang lebih baik ditangani oleh pemanfaatan model NLI pada sistem tanya jawab berbahasa Indonesia.
    
\end{itemize}

%-----------------------------------------------------------------------------%
\section{Sistematika Penulisan}
\label{1.6}
%-----------------------------------------------------------------------------%
Laporan ini diawali dengan abstrak yang terdiri dari ringkasan keseluruhan isi penelitian, pernyataan masalah, metode yang digunakan, analisis dan eksperimen dari hasil penelitian, dan kesimpulan penelitian; daftar isi, daftar gambar, dan daftar tabel. Kemudian, laporan dilanjut ke bab-bab penyusun, antara lain adalah, Bab I yang berisi bagian pendahuluan, Bab II yang berisi bagian studi literatur, Bab III yang berisi bagian metodologi, Bab IV yang berisi implementasi, Bab V yang berisi bagian hasil dan analisis, dan Bab VI yang berisi bagian penutup. Lampiran yang berisi lampiran terkait kode program, penjelasan tambahan, ataupun tabel.

Kemudian, pada Bab I akan membahas hal-hal yang berkaitan tentang latar belakang topik penelitian, rumusan masalah, tujuan penelitian, manfaat penelitian, ruang lingkup penelitian, batasan penelitian, dan sistematika penulisan. Lalu, pada Bab II akan membahas hal-hal yang berkaitan tentang studi literatur yang berkaitan dengan topik penelitian yang juga akan menjadi landasan teori penelitian topik sistem tanya jawab berbahasa Indonesia ini. Kemudian, pada Bab III akan membahas hal-hal yang berkaitan tentang metodologi yang digunakan pada penelitian ini. Lalu, pada Bab IV akan membahas hal-hal yang berkaitan tentang implementasi teknis yang penulis lakukan dari masing-masing eksperimen. Kemudian, pada Bab V akan membahas hal-hal yang berkaitan tentang analisis dan eksperimen dari hasil penelitian yang telah penulis lakukan. Lalu, pada Bab VI akan diberikan kesimpulan dan saran dari penelitian yang telah penulis lakukan terhadap topik sistem tanya jawab berbahasa Indonesia ini. Terakhir, pada Bab Lampiran akan diberikan lampiran terkait kode program, penjelasan tambahan, ataupun tabel.