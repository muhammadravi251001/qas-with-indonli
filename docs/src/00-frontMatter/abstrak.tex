%
% Halaman Abstrak
%
% @author  Andreas Febrian
% @version 2.1.2
% @edit by Ichlasul Affan
%

\chapter*{Abstrak}
\singlespacing

\vspace*{0.2cm}

\noindent \begin{tabular}{l l p{10cm}}
	\ifx\blank\npmDua
		Nama&: & \penulisSatu \\
		Program Studi&: & \programSatu \\
	\else
		Nama Penulis 1 / Program Studi&: & \penulisSatu~/ \programSatu\\
		Nama Penulis 2 / Program Studi&: & \penulisDua~/ \programDua\\
	\fi
	\ifx\blank\npmTiga\else
		Nama Penulis 3 / Program Studi&: & \penulisTiga~/ \programTiga\\
	\fi
	Judul&: & \judul \\
	Pembimbing&: & \pembimbingSatu \\
	\ifx\blank\pembimbingDua
    \else
        \ &\ & \pembimbingDua \\
    \fi
    \ifx\blank\pembimbingTiga
    \else
    	\ &\ & \pembimbingTiga \\
    \fi
\end{tabular} \\

\vspace*{0.5cm}

\noindent Sistem tanya jawab merupakan salah satu tugas dalam domain \emph{natural language processing} (NLP) yang sederhananya bertugas untuk menjawab pertanyaan sesuai konteks yang pengguna berikan ke sistem tanya jawab tersebut. Pada saat ini sistem tanya jawab mulai dibutuhkan secara massal setelah meluasnya penggunaan AI \emph{chatbot} yang mampu menjawab berbagai pertanyaan yang diberikan oleh pengguna. Sistem tanya jawab berbahasa Indonesia sebenarnya sudah ada, namun masih memiliki performa yang terbilang kurang baik. Penelitian ini bereksperimen untuk mencoba meningkatkan performa dari sistem tanya jawab berbahasa Indonesia dengan memanfaatkan \emph{natural language inference} (NLI). Eksperimen untuk meningkatkan sistem tanya jawab berbahasa Indonesia, penulis menggunakan dua metode, yaitu: \emph{intermediate-task transfer learning} dan \emph{task recasting} sebagai verifikator. Dengan metode \emph{intermediate-task transfer learning}, performa sistem tanya jawab berbahasa Indonesia meningkat, hingga skor F1-nya naik sekitar 5.69 dibandingkan tanpa menggunakan pemanfaatan NLI sama sekali, dan berhasil mendapatkan skor F1 tertinggi sebesar 85.14, namun, peningkatan performa dengan metode \emph{intermediate-task transfer learning} cenderung tidak signifikan, kecuali pada beberapa kasus khusus model tertentu. Sedangkan dengan metode \emph{task recasting} sebagai verifikator dengan parameter tipe \emph{filtering} dan tipe perubahan format kalimat, performa sistem tanya jawab berbahasa Indonesia cenderung menurun, penurunan performa ini bervariasi signifikansinya. Pada penelitian ini juga dilakukan analisis karakteristik pasangan konteks-pertanyaan-jawaban seperti apa yang bisa dijawab dengan lebih baik oleh sistem tanya jawab dengan memanfaatkan NLI, dan didapatkan kesimpulan bahwa: performa sistem tanya jawab meningkat pada berbagai karakteristik, antara lain: pada tipe pertanyaan apa, dimana, kapan, siapa, bagaimana, dan lainnya; kemudian pada panjang konteks $\leq100$ dan $101\leq150$; lalu pada panjang pertanyaan $\leq5$ dan $6\leq10$; kemudian pada panjang jawaban \emph{golden truth} $\leq5$ dan $6\leq10$; lalu pada keseluruhan \emph{answer type} selain \emph{law} dan \emph{time}; terakhir pada \emph{reasoning type} WM, SSR, dan MSR. \\

\vspace*{0.2cm}

\noindent Kata kunci: \\ sistem tanya jawab, \emph{natural language inference}, bahasa Indonesia, \emph{intermediate-task transfer learning}, \emph{task recasting} \\

\setstretch{1.4}
\newpage
