%
% Halaman Abstract
%
% @author  Andreas Febrian
% @version 2.1.2
% @edit by Ichlasul Affan
%

\chapter*{ABSTRACT}
\singlespacing

\vspace*{0.2cm}

\noindent \begin{tabular}{l l p{11.0cm}}
	\ifx\blank\npmDua
		Name&: & \penulisSatu \\
		Study Program&: & \studyProgramSatu \\
	\else
		Writer 1 / Study Program&: & \penulisSatu~/ \studyProgramSatu\\
		Writer 2 / Study Program&: & \penulisDua~/ \studyProgramDua\\
	\fi
	\ifx\blank\npmTiga\else
		Writer 3 / Study Program&: & \penulisTiga~/ \studyProgramTiga\\
	\fi
	Title&: & \judulInggris \\
	Counselor&: & \pembimbingSatu \\
	\ifx\blank\pembimbingDua
	\else
		\ &\ & \pembimbingDua \\
	\fi
	\ifx\blank\pembimbingTiga
	\else
		\ &\ & \pembimbingTiga \\
	\fi
\end{tabular} \\

\vspace*{0.5cm}

\noindent The question-answering system is one of the tasks within the domain of natural language processing (NLP) that, in simple terms, aims to answer questions based on the context provided by the user to the question-answering system. While there is an existing Indonesian question-answering system, its performance is considered somewhat inadequate. This research conducts experiments to improve the performance of the Indonesian question-answering system by utilizing natural language inference (NLI). In order to enhance the Indonesian question-answering system, the author employs two methods: intermediate-task transfer learning and task recasting as verifiers. Using the intermediate-task transfer learning method, the performance of the Indonesian question-answering system improves significantly, with an increase of approximately 5.69 in F1 score compared to not utilizing NLI at all, achieving the highest F1 score of 85.14. However, the performance improvement with the intermediate-task transfer learning method tends to be non-significant, except in certain specific cases and particular models. On the other hand, employing the task recasting method as a verifier with filtering parameter type and sentence format change type leads to a decline in the performance of the Indonesian question-answering system, with the significance of this performance decrease varying. Additionally, this research conducts an analysis on the characteristics of context-question-answer pairs that can be better answered by the question-answering system utilizing NLI. The findings conclude that the question-answering system's performance improves compared to its baseline across various characteristics, including different question types such as what, where, when, who, how, and others. Furthermore, it improves with context lengths $\leq100$ and $101\leq150$, question lengths $\leq5$ and $6\leq10$, as well as answer lengths (golden truth) $\leq5$ and $6\leq10$. Additionally, it performs better in overall answer types excluding law and time, and lastly, in reasoning types WM, SSR, and MSR.

\vspace*{0.2cm}

\noindent Key words: \\ question answering system, natural language inference, Indonesian, intermediate-task transfer learning, task recasting, performance improvement, characteristics of context-question-answer pairs. \\

\setstretch{1.4}
\newpage
