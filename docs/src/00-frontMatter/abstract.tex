%
% Halaman Abstract
%
% @author  Andreas Febrian
% @version 2.1.2
% @edit by Ichlasul Affan
%

\chapter*{ABSTRACT}
\singlespacing

\vspace*{0.2cm}

\noindent \begin{tabular}{l l p{11.0cm}}
	\ifx\blank\npmDua
		Name&: & \penulisSatu \\
		Study Program&: & \studyProgramSatu \\
	\else
		Writer 1 / Study Program&: & \penulisSatu~/ \studyProgramSatu\\
		Writer 2 / Study Program&: & \penulisDua~/ \studyProgramDua\\
	\fi
	\ifx\blank\npmTiga\else
		Writer 3 / Study Program&: & \penulisTiga~/ \studyProgramTiga\\
	\fi
	Title&: & \judulInggris \\
	Counselor&: & \pembimbingSatu \\
	\ifx\blank\pembimbingDua
	\else
		\ &\ & \pembimbingDua \\
	\fi
	\ifx\blank\pembimbingTiga
	\else
		\ &\ & \pembimbingTiga \\
	\fi
\end{tabular} \\

\vspace*{0.5cm}

\noindent Question-answering system is one of the tasks in the domain of natural language processing (NLP) that essentially aims to answer questions based on the context provided by the user to the question-answering system. Currently, question-answering systems are in high demand due to the widespread use of AI chatbots capable of answering various user questions. Although there are existing question-answering systems in the Indonesian language, their performance is considered subpar. This research experiment aims to improve the performance of Indonesian question-answering systems by leveraging natural language inference (NLI). In order to enhance the Indonesian question-answering system, the author utilizes two methods: intermediate-task transfer learning and task recasting as verifiers. With the intermediate-task transfer learning method, the performance of the Indonesian question-answering system improves significantly, with an increase of approximately 5.69 in F1 score compared to not utilizing NLI at all, achieving the highest F1 score of 85.14. However, the performance improvement with the intermediate-task transfer learning method tends to be insignificant, except in certain specific cases and particular models. On the other hand, using the task recasting method as a verifier with filtering parameter type and sentence format change type, the performance of the Indonesian question-answering system tends to decrease, with varying degrees of significance in performance decline. This research also analyzes the characteristics of context-question-answer pairs that can be better answered by question-answering systems utilizing NLI. The conclusions drawn from the analysis are as follows: the question-answering system's performance improves across various characteristics, including different question types such as what, where, when, who, how, and others; it also improves with context length $\leq100$ and $101\leq150$; furthermore, with question length $\leq5$ and $6\leq10$; as well as with the answer's golden truth length $\leq5$ and $6\leq10$; it also improves with answer types other than law and time; and finally, with reasoning types WM, SSR, and MSR.

\vspace*{0.2cm}

\noindent Key words: \\ question answering system, natural language inference, Indonesian, intermediate-task transfer learning, task recasting \\

\setstretch{1.4}
\newpage
