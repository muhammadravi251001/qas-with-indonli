%-----------------------------------------------------------------------------%
\chapter*{\kataPengantar}
%-----------------------------------------------------------------------------%
\pagestyle{first-pages}

Puji syukur penulis panjatkan kepada Allah Subhanahu Wa Ta'ala karena atas berkah dan rahmat-Nya, laporan tugas akhir ini dapat penulis selesaikan dengan lancar tanpa ada hambatan yang berarti. Penulisan laporan tugas akhir ini penulis tujukan untuk memenuhi salah satu syarat memperoleh gelar Sarjana Ilmu Komputer pada Fakultas Ilmu Komputer Universitas Indonesia. Karena bantuan dari berbagai pihak secara langsung atau tidak langsung, penulis dapat menyelesaikan laporan ini dengan lancar. Oleh karena itu, penulis ingin mengucapkan terima kasih kepada:

\begin{enumerate}[topsep=0pt,itemsep=-1ex,partopsep=1ex,parsep=1ex]

\item Keluarga penulis, baik Buya, Bunda, Zulfikar, Zulhaydar, dan Kakek penulis yang telah memberikan bantuan, dukungan, do'a, dan semangat moral yang tak terhitung bagi jalannya tugas akhir ini. Bantuan secara langsung atau tidak langsung pun terus-menerus mengalir dari keluarga penulis, sehingga penulis dapat menyelesaikan tugas akhir ini dengan lancar.

\item Kedua dosen pembimbing penulis, Bapak Rahmad Mahendra, S.Kom., M.Sc. dan Bapak Alham Fikri Aji, Ph.D., yang telah membantu dan membimbing penulis bahkan semenjak penelitian tugas akhir ini dimulai sampai saat ini, dengan bantuan dan bimbingan tersebut, penulis dapat menyelesaikan tugas akhir ini dengan lancar.

\item Dosen pembimbing akademik, Ari Saptawijaya, S.Kom., M.Sc., Ph.D., yang telah membantu perkuliahan secara umum selama penulis menimba ilmu di Fakultas Ilmu Komputer Universitas Indonesia.

\item Kepada tim Tokopedia-UI AI \emph{Center of Excellence} yang telah berkenan meminjamkan mesin GPU untuk penulis gunakan dalam penelitian ini.

\item Rekan dekat penulis, Muhamad Fauzi Ridwan, yang telah membantu penulis pada masa-masa terburuk akademik penulis saat semester-semester awal hingga menengah. Sedikit banyaknya, bantuan pengajaran dan semangat moral dari Fauzi, membuat penulis tetap semangat belajar dan mencoba bangkit dari keterpurukan akademik yang membebani pikiran penulis.

\item Berbagai pihak lain yang sulit untuk disebutkan satu per satu yang telah membantu penulis dalam proses perkuliahan maupun penyelesaian tugas akhir.

\end{enumerate}

Penulis menyadari bahwa laporan tugas akhir ini masih jauh dari sempurna. Oleh karena itu, apabila terdapat kesalahan, kekurangan, atau koreksi dalam laporan tugas akhir ini, Penulis memohon agar kritik dan saran bisa disampaikan langsung melalui \f{e-mail} \code{muhammadravi251001@gmail.com}.

% Untuk input gambar tanda tangan, silahkan sesuaikan xshift, yshift, dan width dengan gambar tanda tangan Anda
%\begin{tikzpicture}[remember picture,overlay,shift={(current page.north east)}]
%\node[anchor=north east,xshift=-3cm,yshift=-6.2cm]{\includegraphics[width=3cm]{assets/pics/tanda_tangan_wikipedia.png}};
%\end{tikzpicture}

\vspace*{0.1cm}
\begin{flushright}
Depok, \tanggalSiapSidang\\[0.1cm]
\ifx\blank\npmDua
	\vspace*{1.5cm}
	\penulisSatu
\else
	Tim Penulis
\fi

\end{flushright}
